\documentclass[conference,final]{IEEEtran}
\IEEEoverridecommandlockouts
% The preceding line is only needed to identify funding in the first footnote. If that is unneeded, please comment it out.
\usepackage{cite}
\usepackage{hyperref}
\usepackage{amsmath,amssymb,amsfonts}
\usepackage{algorithmic}
\usepackage{graphicx}
\usepackage{float}
\usepackage{array}
\usepackage{textcomp}
\usepackage{xcolor}
\usepackage{amsmath}
\usepackage[nolist,nohyperlinks]{acronym}
\def\BibTeX{{\rm B\kern-.05em{\sc i\kern-.025em b}\kern-.08em
    T\kern-.1667em\lower.7ex\hbox{E}\kern-.125emX}}
\begin{document}

\title{Quadruped Locomotion with Central Pattern Generators
and Deep Reinforcement Learning\\
{\footnotesize Legged Robots MICRO-507 - Miniproject 2}
}

\author{\IEEEauthorblockN{Lukas W. Falk Nyholm}
\IEEEauthorblockA{\textit{Microengineering} \\
\textit{EPFL}\\
Lausanne, Switzerland \\
lukas.nyholm@epfl.ch}
\and
\IEEEauthorblockN{Chelsea Davidson}
\IEEEauthorblockA{\textit{Microengineering} \\
\textit{EPFL}\\
Lausanne, Switzerland \\
email address or ORCID}
\and
\IEEEauthorblockN {Frederik Steinmetz}
\IEEEauthorblockA{\textit{Mechanical Engineering} \\
\textit{EPFL}\\
Lausanne, Switzerland \\
frederik.steinmetz@epfl.ch}
}
%information
%The target report length is 6-8 pages in double-column format, excluding references and generative AI declaration and supplementary material. For supplementary material, we allow 3 additional pages in case you have extra diagrams to show but the main plots and graphs should be in the main report. We leave it to your creativity on how you want to organise your report!

\begin{acronym}
\acro{cpg}[CPG]{central pattern generator}
\acro{rl} [RL] {reinforcement learning}
\end{acronym}

\maketitle

\section{Introduction}

Quadruped locomotion remains a central challenge in legged robotics due to the complex interaction between high-dimensional dynamics, intermittent ground contact, and the need for robustness across varying speeds and terrains. Two major paradigms have emerged to address this problem: biologically inspired controllers such as Central Pattern Generators (CPGs), and data-driven approaches based on deep reinforcement learning (DRL).

In this project, we investigate and compare these paradigms for quadruped locomotion using a simulated A1-like robot in PyBullet. In the first part, we design open-loop locomotion controllers based on coupled Hopf oscillators, mapping rhythmic CPG outputs to Cartesian foot trajectories and tracking them using joint-space and Cartesian-space proportional–derivative (PD) control. Multiple gaits are implemented, with a focus on stable trot locomotion across a range of speeds.

In the second part, we design a Markov Decision Process (MDP) and train locomotion policies using deep reinforcement learning, following the CPG-RL framework proposed by Bellegarda and Ijspeert \cite{bellegarda2022cpg}. Reinforcement learning is used to modulate CPG parameters rather than directly commanding joints, combining the inherent stability of rhythmic control with the adaptability of learning-based methods. Both velocity-tracking and task-specific controllers are trained and evaluated.

The contributions of this report are threefold:
\begin{itemize}
    \item A systematic evaluation of CPG-based locomotion with joint and Cartesian PD tracking
    \item A comparison of observation and action space design choices for CPG-RL
    \item An analysis of robustness, performance, and extensibility for both classical and learning-based controllers
\end{itemize}



\section{Methodology} \label{methodology}
In this section, the methodology is presented that was used to create the policy for reinforcement learning of the quadruped. First, the \ac{cpg} needs to be implemented. This is done in the \texttt{run\_cpg.py}-file. The presentation of the approach follows the order of the functions in the code-file. First, we altered the pd gains for the Cartesian system, $k_p$ was set to $k_p = \begin{bmatrix}
1200 & 2000 & 1200
\end{bmatrix}^\intercal$
 and $k_d$ was set to $k_d=
\begin{bmatrix}
45 & 20 & 45
\end{bmatrix}^\intercal$.
It shows that that proportional gain is two orders of magnitude larger than the derivative one. Afterwards, the desired foot position for each leg is determined, consisting of the $x$, $y$, and $z$ coordinates. The left and right feet are distinguished by the sign function to account for the dimension of the hip.  Thereafter the desired foot velocity is obtained from the function
\texttt{env.robot.ComputeInverseKinematics(i, leg\_xyz\_des)}.
Then, the jacobian, position and velocity in leg frame are calculated if the the contribution of the cartesian is set to true. In this part the actual and desired feet positions as well as joint angles are set. Afterwards, the \ac{cpg}-states, the amplitude $r$ and the phase $\theta$ are saved as well as their corresponding derivatives $\dot{r}$ and $\dot{\theta}$.
Once the state data has been collected, the plots can be genereated, depending on which options are enabled. For \ac{cpg} plotting, the code produces four time-series plots showing the four above mentioned \ac{cpg} variables of the chosen leg with the phase limited to the interval $\left[ - \pi, \pi \right]$. The plots can be used to help debugging the code as it shows if the parameters are displayed as expected.
Furthermore, the quality of the resulting locomotion gets evaluated. This is achieved by plotting the actual and desired foot trajectories in Cartesian space for $x$, $y$, and $z$ coordinates each enabling a precise monitoring of the targeted foot motion over time. It not only shows the difference between actual and desired position but also give insight about the general motion smoothness. 
Finally, the code is implemented to plot the joint tracking performance. If activated, the joint angles for one of four legs are plotted, capturing hip, thigh, and calf joints for time. Again, the actual and desired angles are recorded simultaneously. That provides the same insights as described for the foot placement above.
The plotting is crucial as it provides insights into the overall \ac{cpg} performance. As this is the foundation for the \ac{rl} part, the plots enables us to verify whether the \ac{cpg} is corretly implemented.

\begin{enumerate}
    % Plot of CPG states
    \item 
    % A plot comparing the desired foot position vs actual foot position
    \item
    % A plot comparing the desired joint angles vs actual joint angles
    \item 
    % Hyperparameters: For hyperparameters we changed the cartesin Kp and Kd contributions. We handtuned them from being uniform, to improve the tracking for y-movement, so it doesn't veer to much to the sides.
    \item 
    % Videos
    \item 
    % Discussion on feedback loop
    \item 
\end{enumerate}


\section{Results}

\subsection{CPG State Evolution}

Fig.~\ref{fig:cpg_states} shows the CPG states for a trot gait over two gait cycles. Diagonal leg pairs (FR--RL and FL--RR) evolve in phase, with a $\pi$ phase offset between pairs. The oscillator amplitude converges smoothly to $\sqrt{\mu}$, and the phase derivative switches cleanly between swing and stance frequencies.

\begin{figure}[H]
    \centering
    \includegraphics[width=0.5\textwidth]{images/CPG_States_TROT_unwrapped.png}
    \caption{CPG states for Trot Gait}
    \label{fig:cpg_states}
\end{figure}

\subsection{Tracking Performance}
To perform both feet and joint tracking, Joint and Cartesian PD was used with the following gains:
\begin{itemize}
    \item \textbf{$K_{p,Joint}$}: [100,100,100]
    \item \textbf{$K_{d,Joint}$}: [2, 2, 2]
    \item \textbf{$K_{p,Cartesian}$}: diag(1200, 2000, 1200)
    \item \textbf{$K_{d,Cartesian}$}: diag(45, 20, 45)
\end{itemize}

Moderate joint PD gains provide baseline stability, while Cartesian PD gains dominate tracking performance; appropriately tuned task-space stiffness improves both foot and joint tracking, whereas excessive Cartesian gains degrade performance due to over-constrained contact dynamics.


\paragraph{Foot Tracking Performance}
Using joint-space PD control alone results in poor Cartesian foot tracking and causes the robot to collapse (see Figure ~\ref{fig:footTrack_jointPD}). This is expected, as joint PD does not directly regulate foot position or compensate for contact forces.

\begin{figure}[H]
    \centering
    \includegraphics[width=0.5\textwidth]{images/CPG_desiredVsActualFeetPos_JustJointPD.png}
    \caption{Foot Tracking performance using Joint PD only}
    \label{fig:footTrack_jointPD}
\end{figure}


Adding Cartesian PD dramatically improves tracking performance (see Figure ~\ref{fig:footTrack_cartPD}). The actual foot trajectory closely follows the desired path in all directions, particularly in the sagittal plane. Increasing the Cartesian lateral stiffness beyond nominal values degraded tracking due to over-constraining the contact dynamics (see Figure ~\ref{fig:footTrack_cartPD_high}).

\begin{figure}[H]
    \centering
    \includegraphics[width=0.5\textwidth]{images/CPG_desiredVsActualFeetPos_JointAndCartesianPD.png}
    \caption{Foot Tracking performance using Cartesian and Joint PD}
    \label{fig:footTrack_cartPD}
\end{figure}

\paragraph{Joint Tracking Performance}

Joint-space tracking performance improves significantly when Cartesian PD feedback is added on top of the joint-level controller. With only joint PD, noticeable tracking errors are observed as external contact forces and unmodeled dynamics introduce disturbances that cannot be fully compensated by independent joint-space feedback. This leads to deviations from the desired joint trajectories.

By introducing Cartesian PD control, these disturbances are indirectly regulated through task-space constraints. The Cartesian controller enforces accurate foot position and velocity tracking, which in turn constrains the joint motion through the robot’s kinematic structure. As a result, joint trajectories become more consistent and exhibit reduced oscillations, even though the Cartesian controller does not explicitly act in joint space.

\begin{figure}[H]
    \centering
    \includegraphics[width=0.5\textwidth]{images/CPG_desiredVsActualJointPos_JustJointPD.png}
    \caption{Joint Tracking performance using Joint PD only}
    \label{fig:jointTrack_jointPD}
\end{figure}

\begin{figure}[H]
    \centering
    \includegraphics[width=0.5\textwidth]{images/CPG_desiredVsActualJointPos_JointAndCartesianPD_BETTER.png}
    \caption{Joint Tracking performance using Joint and Cartesian PD}
    \label{fig:jointTrack_CartPD}
\end{figure}

\subsection{Locomotion Speed, Gait Timing, and Efficiency}

\paragraph{Key Hyperparameters Tuned}
In order to achieve stable gaits for different locomotion speeds, the following parameters were necessary to tune. The sets shown express the values required to achieve slow, nominal, and fast trotting gaits respectively. 
\begin{table}[h]
\centering
\caption{Tuned CPG and Cartesian control parameters for slow, nominal, and fast trotting gaits.}
\begin{tabular}{l c c c >{\raggedright\arraybackslash}p{2.6cm}}
\hline
\textbf{Parameter} & \textbf{Slow} & \textbf{Nominal} & \textbf{Fast} & \textbf{Role} \\
\hline
Swing frequency $\omega_{\text{swing}}$ (rad/s) & $1.5\cdot2\pi$ & $5\cdot2\pi$ & $9\cdot2\pi$ & Sets swing cadence \\
Stance frequency $\omega_{\text{stance}}$ (rad/s) & $0.5\cdot2\pi$ & $2\cdot2\pi$ & $4\cdot2\pi$ & Controls duty cycle \\
Step length $\ell_{\text{step}}$ (m) & 0.015 & 0.05 & 0.09 & Determines forward speed \\
Ground clearance $h_{\text{clear}}$ (m) & 0.05 & 0.07 & 0.08 & Avoids foot scuffing \\
Ground penetration $d_{\text{pen}}$ (m) & 0.003 & 0.01 & 0.02 & Contact compliance \\
PD gain scale $\alpha_{p,d}$ & 0.7 & 1.0 & 1.5 & Cartesian stiffness and damping \\
\hline
\end{tabular}
\end{table}



\paragraph{Gait Analysis Terms}
\begin{table}[H]
\centering
\caption{Gait timing and performance metrics.}
\begin{tabular}{l l}
\hline
\textbf{Quantity} & \textbf{Definition} \\
\hline
Swing duration $T_{\text{swing}}$ & $\displaystyle \frac{\pi}{\omega_{\text{swing}}}$ \\
Stance duration $T_{\text{stance}}$ & $\displaystyle \frac{\pi}{\omega_{\text{stance}}}$ \\
Step duration $T_{\text{step}}$ & $T_{\text{swing}} + T_{\text{stance}}$ \\
Duty cycle $D$ & $\displaystyle \frac{T_{\text{stance}}}{T_{\text{step}}}$ \\
Cost of Transport (CoT) & $\displaystyle \frac{\int_0^T \sum_i |\tau_i \dot{q}_i|\,dt}{mgd}$ \\
\hline
\end{tabular}
\end{table}

\paragraph{Achieved Locomotion Regimes}
\begin{itemize}
    \item \textbf{Slow trot}
    \\Refer to Video TROT\_LOW\_0.025ms.mp4
    \begin{itemize}
        \item Forward velocity: $v = 0.025\,\mathrm{m/s}$
        \item Swing duration: $T_{\text{swing}} = \text{0.333}\,\mathrm{s}$
        \item Stance duration: $T_{\text{stance}} = \text{1.0}\,\mathrm{s}$
        \item Step duration: $T_{\text{step}} = \text{1.3}\,\mathrm{s}$
        \item Duty cycle: $D = \text{0.7}$
        \item Cost of Transport: $\mathrm{CoT} = 4.325$
    \end{itemize}

    \item \textbf{Nominal trot}
    \\Refer to Video TROT\_NOMINAL\_0.384.mp4
    \begin{itemize}
        \item Forward velocity: $v = 0.384\,\mathrm{m/s}$
        \item Swing duration: $T_{\text{swing}} = \text{0.1}\,\mathrm{s}$
        \item Stance duration: $T_{\text{stance}} = \text{0.25}\,\mathrm{s}$
        \item Step duration: $T_{\text{step}} = \text{1.3}\,\mathrm{s}$
        \item Duty cycle: $D = \text{0.71}$
        \item Cost of Transport: $\mathrm{CoT} = 0.865$
    \end{itemize}

    \item \textbf{Fast trot}
    \\Refer to Video TROT\_HIGH\_1.483ms.mp4
    \begin{itemize}
        \item Forward velocity: $v = 1.483\,\mathrm{m/s}$
        \item Swing duration: $T_{\text{swing}} = \text{0.0556}\,\mathrm{s}$
        \item Stance duration: $T_{\text{stance}} = \text{0.125}\,\mathrm{s}$
        \item Step duration: $T_{\text{step}} = \text{1.806}\,\mathrm{s}$
        \item Duty cycle: $D = \text{0.69}$
        \item Cost of Transport: $\mathrm{CoT} = 1.082$
    \end{itemize}
\end{itemize}
Very slow gaits exhibited high CoT due to prolonged stance phases and low dynamic efficiency. Moderate-speed trots achieved the lowest CoT by balancing dynamic motion with stable foot contact.

\subsection{Reinforcement Learning Results}

\paragraph{Observation Space}

As shown in the Figures \ref{fig:min_obs}, \ref{fig:med_obs}, \ref{fig:full_obs} and Videos, the minimal observation space performed the worst, exhibiting the largest velocity tracking errors, a tentative gait, and occasional failure during early execution. While additional training could potentially improve performance, the paper highlighted that this task is only suited to move forward at a particular desired velocity. Therefore, the minimal observation space appears unsuitable for this project.

In contrast, the full and medium observation spaces yielded very similar performance in both the quantitative metrics and qualitative gait behavior. ~\cite{bellegarda2022cpg} reports that the medium observation space is expected to be more robust, as it excludes joint-level sensory information that can be noisy on real hardware and may unnecessarily increase policy sensitivity. Furthermore, since the inclusion of additional observations in the full observation space did not result in noticeable performance gains, the increased computational and sensory complexity is not justified. Therefore, the medium observation space was selected for both the final velocity and slope controller.

Including the linear and angular velocities enable accurate velocity tracking and balance control. Base orientation informs the policy of body posture, allowing correction of pitch and roll disturbances. Joint positions and velocities are essential for coordinating the CPG outputs with the robot’s actual limb configuration and dynamics. CPG phase information is provided to synchronize learned residual actions with the underlying rhythmic gait structure. Finally, commanded velocity inputs are included to condition the policy on the desired motion objective.


\paragraph{Action Space}
As seen in Figures \ref{fig:cpg_action} and \ref{fig:cartPD_action} and their corresponding videos, the CPG action space produced the most natural locomotion with superior velocity tracking. In contrast, the Cartesian PD controller caused the robot to appear as if it were “tiptoeing” rapidly. This behavior arises because the controller focuses solely on foot positions in the Cartesian frame; at 1m/s, the learned policy favors small, fast movements to achieve the target velocity. While incorporating joint positions and velocities and constraining joint speeds could encourage larger, more natural strides, this was not possible with the medium observation space.
\begin{figure}[H]
    \centering
    \includegraphics[width=0.5\textwidth]{images/Cartesian_PD_Velocity.png}
    \caption{Cartesian PD Velocity Tracking}
    \label{fig:cartPD_action}
\end{figure}


Comparing the training plots in \ref{fig:cartPD_action_train}, \ref{fig:cartPD_action_train_reward} and \ref{fig:cpg_action_train}, \ref{fig:CPG_action_train_reward} the Cartesian PD controller exhibits a steeper learning curve, as the action space is inherently less stable and early training focuses on achieving basic stability. In contrast, the CPG action space is more stable by design, resulting in a shallower learning curve and allowing training to focus on optimizing performance rather than merely maintaining balance. Additionally, the CPG requires fewer parameters to learn, further contributing to efficient training.

\begin{figure}[H]
    \centering
    \includegraphics[width=0.5\textwidth]{images/CartesianPD_PPO_Ep_Len.png}
    \caption{Cartesian PD Training Episode Length}
    \label{fig:cartPD_action_train}
\end{figure}

\begin{figure}[H]
    \centering
    \includegraphics[width=0.5\textwidth]{images/CPG_PPO_Ep_Len.png}
    \caption{CPG Training Episode Length}
    \label{fig:cpg_action_train}
\end{figure}

\paragraph{Impact of Environment Details}
The effect of curriculum learning can be seen in Figure \ref{fig:no_curriculum} which performed considerably worse than \ref{fig:slope_controller}. The effect of domain randomisation can be seen in Figure \ref{fig:slope_different_height} where the controller was able to climb up an unseen incline level. While these environmental additions improved robustness, increasing the noise from 0.01 to 0.04, significantly degraded the training performance. The higher noise magnitude disrupted the policy’s ability to accurately estimate the robot’s state, leading to unstable learning dynamics and poorer locomotion performance. This indicated that excessive noise overwhelmed the learning signal. This result highlights a trade-off between robustness and learnability: while moderate noise can improve generalization, overly aggressive noise injection can hinder training and reduce final policy performance.







\section{Discussion}
\subsection{CPG Locomotion Controller}
\paragraph{Joint and Cartesian PD Control}
Cartesian PD control is essential for stabilizing CPG-based locomotion, particularly during stance. While joint PD alone is insufficient, task-space feedback allows the controller to generate stabilizing ground reaction forces.

\paragraph{Potential Extension of Controller}
The current controller combines an open-loop Hopf CPG with joint-space and Cartesian PD tracking, which enables stable periodic locomotion but limits adaptability. This framework can be extended through sensory feedback and descending control signals to support more versatile behaviors.

Low-level feedback such as body velocity, foot contact, and body orientation can be used to modulate CPG frequency, phase, and leg amplitude online, improving speed regulation, contact synchronization, and balance. Descending control signals allow higher-level modulation of locomotion without altering the oscillator structure: forward speed can be adjusted via step length and frequency, heading via asymmetric leg modulation, and gait transitions via changes in oscillator phase coupling. Foot placement can be refined by adding task-space offsets to the Cartesian trajectories.

A hierarchical control structure could further integrate a high-level planner or learning-based policy that provides descending commands (e.g., speed, gait, heading), while the CPG and PD controllers handle low-level rhythm generation and tracking, mirroring biological locomotion.


\subsection{Reinforcement Learning}
\paragraph{Robustness}
Our approach shows good robustness in simulation due to the use of curriculum learning, command and terrain randomization, and the inclusion of observation and actuation noise during training (noise level of 0.1). 
The use of a CPG-based controller further improves robustness by enforcing rhythmic, smooth locomotion patterns. Since the policy modulates CPG parameters instead of directly commanding joint torques or positions, the resulting motions are more regular and less sensitive to small disturbances, delays, or modeling errors. This structure also reduces the effective control complexity, which helps stabilize learning and execution.

\paragraph{Sim-to-Sim}
Sim-to-sim transfer is expected to work reasonably well if the robot model and control interface are similar. However, differences in physics engines, contact modeling, friction, timestep size, and actuator dynamics can still degrade performance.

\paragraph{Sim-to-Real}
Sim-to-real transfer is more challenging. Potential issues include unmodeled actuator dynamics, torque and velocity limits, sensor noise, latency, inaccurate mass and inertia parameters, and differences in ground contact behavior. Although training with noise and using a CPG-based policy improves robustness, additional domain randomization including adding/subtracting mass/inertia from various links, varying the coefficient of friction, varying the terrain more (add small boxes etc.), and increasing the noise would improve robustness when transferring this policy to hardware.



\section{Conclusion}

This project investigated quadruped locomotion using both classical CPG-based control and deep reinforcement learning. CPGs combined with Cartesian PD control enabled stable and efficient locomotion across a wide speed range. Reinforcement learning further improved adaptability by modulating CPG parameters, producing smooth and robust behaviors.

The results demonstrate that combining biologically inspired structure with learning-based optimization yields controllers that are both stable and flexible. Future work includes extending the framework to turning, gait transitions, and real-world deployment through more extensive domain randomization and hardware-aware modeling.


\section{Appendix}
\begin{figure}[H]
    \centering
    \includegraphics[width=0.5\textwidth]{images/CPG_desiredVsActualFeetPos_JointAndCartesianPD_1200-5000-1200_45-50-45.png}
    \caption{Foot Tracking performance using Cartesian and Joint PD with increased Cartesian PD Gains}
    \label{fig:footTrack_cartPD_high}
\end{figure}

\begin{figure}[H]
    \centering
    \includegraphics[width=0.5\textwidth]{images/MinimalObsGood.png}
    \caption{Velocity Tracking Error with Minimal Observation}
    \label{fig:min_obs}
\end{figure}

\begin{figure}[H]
    \centering
    \includegraphics[width=0.5\textwidth]{images/CartesianPD_PPO_Rewards.png}
    \caption{Cartesian PD Training Rewards}
    \label{fig:cartPD_action_train_reward}
\end{figure}

\begin{figure}[H]
    \centering
    \includegraphics[width=0.5\textwidth]{images/CPG_PPO_Rewards.png}
    \caption{CPG Training Rewards}
    \label{fig:CPG_action_train_reward}
\end{figure}




\input{sections/acknowledgement}

% References
\bibliographystyle{IEEEtran}
\bibliography{bibliography}

\input{sections/AI_declaration}



\end{document}
